\documentclass[11pt,english,a4paper]{article}
\usepackage[utf8]{inputenc}
\usepackage{babel,csquotes}
\usepackage[T1]{fontenc}
\usepackage[hyphens]{url} % deler lange url-er

% biblatex-pakka med opsjoner
\usepackage[backend=biber,%
style=numeric,%
backref,%
hyperref=true,%
%style=alphabetic,%
%sorting=none,% eller nty, nyt o.a
%date=long,% eller short, terse, comp, iso8601
]{biblatex}

% må opplyse om bib-filen
% ved flere bib-filer gjentas kommandoen
\addbibresource{referanser.bib}

% noen mulige lokale biblatex tilpasninger
\DefineBibliographyStrings{norsk}{%
   urlseen={Sett: },
   bibliography = {Bibliografi},
   references = {Referanser},
   editor = {redaktør},
   translator={oversetter},
   %page={side},
   %pages={sidene},
   and={og},
}

\DeclareFieldFormat{url}{\url{#1}} % fjerner hardkodet "URL: " foran url

\DeclareUrlCommand\url{\def\UrlLeft{\newline}\def\UrlRight{\newline}%
\urlstyle{sf}} % setter inn passende linjeskift

% biblatex anbefaler at hyperref blir lastet inn etter biblatex
\usepackage{hyperref}

\title{User interface}
% Sett inn ditt eget navn her:
\author{}

\begin{document}
\maketitle{}

\tableofcontents

\section{Design of a user interface}
Design of a user interface begins with the task analysis --- an
understanding of the user's underlying tasks and the problem 
domain **Shneiderman,1992**. The user interface should be 
designed in terms of the users' terminology and conception of 
their jobs, rather than the programmer's **Shneiderman, 1983**.

\section{Syntactic level of design: interaction styles}
The principal classes of user interfaces currently in use are
command languages, menus, forms, natural language, direct manipulation,
virtual reality, and combinations of these **Hartson, 1989**. Each
interaction style has its merits for particular user communities or
set of tasks **Myers,1995**.

\section{Command language}
Command language interfaces (CLIs) use artificial languages, 
much like programming languages. They are concise and unambiguous, 
but they are often difficult for a novice to learn and remember 
**Stephenson, 1999**.

\section{Menu}
Menubased user interfaces explicitly present the options available to
a user at each point in a dialogue **Stephenson,1999 and Hutchins,
1986**. 

\section{Natural language}
The principal benefit of natural languages is, of course, that the
user already knows the language **Foley, 1987**. 

\section{Graphical user interface}
In a graphical user interface (GUI), a set of objects called icons is
presented on a scereen, and the user has a repertoire of manipulations
that can be performed on any of them **Jacob, 1986 and Foley, 1990 and
Johnson, 1989**. This means that the user has no command language to
remember beyond the standard set of manipulations, few cognitive
changes of mode, and a reminder of the available objects and their
states shown continuously on the display.

\section{User interface management systems}
A user interface management system (UIMS) is a software 
component that is separate from the application program 
that performs the underlaying task **Olsen, 1992**.

\newpage
% overskriften på referanselista (dokumentklassen article)
\renewcommand{\refname}{Litteraturliste}

% redefiner \bibname ved bruk av dokumentklassen book

% Litteraturlista inn i innholdsfortegnelsen
\printbibliography\addcontentsline{toc}{section}{\refname}

% Eksempel på en referanseliste oppdelt etter dokumenttype.
% Sett et kommentar tegn foran printbibliography-kommandoen
% ovenfor og fjern kommentar-tegnet først på linjene nedenfor.
% Kjør latex-prosessen om igjen.

% \printbibheading
% \printbibliography[type=book, title={Bøker}]
% \printbibliography[type=article, title={Artikler}]
% \printbibliography[type=manual, title={Manualer}]
% \printbibliography[nottype=book, nottype=article,%
% nottype=manual, title={Øvrige dokumenter}]
\end{document}
